\documentclass[12pt]{article}
\usepackage{geometry}
\usepackage{helvet}

\renewcommand{\familydefault}{\sfdefault}
\geometry{letterpaper, portrait, margin=0.75in}
\title{Physics III: Chapter 33 Lecture Notes}
\author{Jake Kratt}


\begin{document}
\maketitle

\section*{Maxwell's Equations}
Maxwell's Equations are a set of equations that govern electromagnetism.
\subsection*{Gauss' Law}
    \[\oint \vec{E}\cdot  d\vec{A} = \frac{q}{\epsilon_{0}}\]
Gauss' Law states that

\subsection*{Gauss' Law (for magnetism)}
\[\oint \vec{B} \cdot d\vec{A}=0\]

\section*{Other Important Equations}


\section*{Practice Problem 1}
Your eyes are the most sensitive to the green-yellow region of the visible light spectrum. If the wavelength of this color is 555nm, calculate the frequency of this light.
\subsection*{Strategy}
Before we get started on solving the problem, we should first define our variables. From the prompt, we can gather the following:
\begin{itemize}
    \item First, let's think about what this problem is asking us for. It is asking us to find the frequency of the light-wave using the wavelength. Shouldn't be too difficult.
    \item One equation that we notice could be really useful in this situation is the equation \[v=f \lambda. \] We already have the wavelength \(\lambda = 555nm\), so this equation could be a good starting point.
    \item Since light travels at the speed of, well, light, we can infer that \(v=c\). Looking back at the equation \(v=f \lambda \), we can rewrite it as \(c = f \lambda\). We conveniently have two of the three values in the equation. So, now we pretty much have everything we need to solve the problem, so let's do that
\end{itemize}
\subsection*{Solution}
Now that we've examined the problem, we have a decent amount of information to actually solve it. I recommend rearranging your equations symbolically, so that's what we'll do here.
\begin{itemize}
    \item Using a little algebra magic, we can rearrange the equation from earlier to get us what we're looking for--the frequency \(f\), by dividing both sides by \(\lambda\), we get \[f = \frac{c}{\lambda}.\]
    \item So, by plugging in our values, we get a final solution of \[f = \frac{3\times10^{8}}{555\times10^{-9}} = 5.41\times10^{14}Hz.\]
\end{itemize}

\end{document}
