\documentclass[12pt, letterpaper]{article}
\usepackage{helvet}
\usepackage{amsmath}
\usepackage{graphicx}
\usepackage{tabularx}
\usepackage{geometry}

% \graphicspath{dir-list}
\geometry{margin = 0.75in}
\renewcommand{\familydefault}{\sfdefault}

\title{Physics III: Chapter 35 - Image Formation}
\author{Jake Kratt}
\date{February 5, 2025}

\begin{document}
\maketitle

\section*{35.1. Images Formed by Flat Mirrors}
The most obvious example of an image formed by a flat mirror would be something like your bathroom mirror. You see it every day (hopefully), and it reflects an image of yourself back at you, though it's flipped. The distance $p$ is the \textbf{object distance}.

\end{document}