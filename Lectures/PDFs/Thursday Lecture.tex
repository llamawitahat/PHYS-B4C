\documentclass[12pt, letterpaper]{article}
\usepackage{helvet}
\usepackage{amsmath}
\usepackage{graphicx}
\usepackage{tabularx}
\usepackage{geometry}

% \graphicspath{dir-list}
\geometry{margin = 0.75in}
\renewcommand{\familydefault}{\sfdefault}


\title{Physics III Lecture: Thin Lenses}
\author{Jake Kratt}
\date{February 6, 2025}

\begin{document}
\maketitle
\section*{Spherical Refracting Surfaces}
Where your image is going to be located based on Snell's law for a spherically refracting surface
\[\frac{n_{1}}{p}+\frac{n_{2}}{q}=\frac{n_{2}-n_{1}}{R}\]

\[M = -\frac{n_{1}q}{n_{2}p}\]

\begin{tabular}{ |c|c|c| }
\hline
Quantity & Positive When... & Negative When... \\
\hline
Object Location (p) & object is in front of surface & object is in back of surface \\
& (real object) & (virtual object) \\
\hline
Image Location (q) & image is in the back of surface & image is in front of surface  \\
& (real image) & (virtual object) \\
\hline
Image height $h'$ & image is upright. & image is inverted. \\
\hline
Radius (R) & center of curvature is  & center of curvature is  \\
& in the back of surface (convex) & in front of the surface (concave) \\
\hline
    
\end{tabular}

\section*{Practice Problem 1}

\subsection*{Important equations}
\[\frac{n_{1}}{p}+\frac{n_{2}}{q}=\frac{n_{2}-n_{1}}{R}\]

\subsection*{Actually solving it}

\begin{align*}
    & \frac{n_{2}}{q} = \frac{n_{2} - n_{1}}{R} - \frac{n_{1}}{p} \\
    & q = \frac{n_{2}}{\frac{n_{2} - n_{1}}{R} - \frac{n_{1}}{p}} \\
    & q = -8.35cm
\end{align*}

\section*{Converging Lenses}
Fill in images

\section*{Diverging Lenses}
Fill in images

\section*{Thin Lenses}

\begin{itemize}
    \item Thin lens equation \[\frac{1}{p} + \frac{1}{q} = \frac{1}{f}\]
    \item Lens-Maker's equation \[\frac{1}{f} = (n-1)\Big(\frac{1}{R_{1}}-\frac{1}{R_{2}}\Big)\]
    \item Magnification equation: \[M = \frac{h'}{h} = -\frac{q}{p}\]
    \item Mathematically, mirrors and lenses are the same thing
\end{itemize}

\section*{Sign Conventions for Thin Lenses}

\begin{tabular}{|c|c|c|}
    \hline
    Quantity & Positive When... & Negative When... \\
    \hline
    Image height ($h'$) & image is upright & center of curvature is in front of lens \\
    \hline
    $R_{1}$ and $R_{2}$ & center of curvature is in back of lens &  \\
    \hline
    
\end{tabular}

\section*{Practice Problem 2}
\subsection*{The approach}
\begin{itemize}
    \item $R_{1}$ is positive
    \item $R_{2}$ is negative
\end{itemize}
\begin{align*}
    & \frac{1}{f} = (n-1)\Big(\frac{1}{R_{1}}-\frac{1}{R_{2}}\Big) \\
    & f = \frac{1}{(n-1)\Big(\frac{1}{R_{1}}-\frac{1}{R_{2}}\Big)} \\
    & f = \frac{1}{(1.50-1)\Big(\frac{1}{2.00cm}-\frac{1}{2.50cm}\Big)} = 20cm
\end{align*}

\section*{Practice Problem 3}

\subsection*{The approach}
\begin{align*}
    & d = 180cm \\
    & M = -5 \\
    & \frac{1}{p} + \frac{1}{q} = \frac{1}{f} \\
    & M = -\frac{q}{p} \implies q = -Mp = 5p \\
    & p + q = 180cm \implies p + 5p = 180cm \\
    & p = 30cm
\end{align*}
We have \(p\), but that hasn't answered the question. However, we do have everything we need to solve for \(f\).
\begin{itemize}
    \item Given: \[f = \frac{1}{(n-1)\Big(\frac{1}{R_{1}}-\frac{1}{R_{2}}\Big)}\]
    \item We can find: \[f = +25cm\]
\end{itemize}

\section*{Practice Problem 4 *WILL BE ON EXAM*}
\end{document}